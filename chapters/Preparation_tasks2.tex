\documentclass[docmute]{article}

% AMS packages for enhanced math typesetting and symbols:
\usepackage{amsmath}  % Provides enhanced math features like align, gather, etc.
\usepackage{amssymb}  % Provides additional math symbols
\usepackage{amsthm}   % Enables theorem-like environments

\usepackage{graphicx} % Provides commands for including and managing external graphics (e.g., figures, images)

% Package for customizing list environments:
\usepackage{enumitem} % Allows control over layout of lists (itemize, enumerate, etc.)

% Full-page layout package:
\usepackage{fullpage} % Uses more of the page area by reducing margins

% TikZ package for drawing graphics:
\usepackage{tikz}     % Used for creating high-quality diagrams and figures

% Microtype package for typographical enhancements:
\usepackage{microtype} % Improves justification, kerning, and overall appearance

% Package for typesetting polynomials:
\usepackage{polynom}  % Provides commands for polynomial long division and related tasks

% Package for controlling figure placement:
\usepackage{placeins} % Provides the \FloatBarrier command to control floating environments

% Forest package for drawing trees:
\usepackage{forest}   % Simplifies the creation of tree diagrams

% Package to allow one LaTeX file to input another:
\usepackage{docmute}  % Allows this file to be included in another document without reloading the preamble

% Load additional TikZ libraries:
\usetikzlibrary{trees} % Provides additional tree-specific commands for TikZ

% Define theorem-like environments using amsthm:
\newtheorem{corollary}{Corollary} % Defines a new "corollary" environment
\newtheorem{lemma}{Lemma}         % Defines a new "lemma" environment


\input{watermark/watermark.tex}


\title{Preparation Tasks 2}
\author{Tomasz Brengos \\  
Committers : Aliaksei Kudzelka}
\date{}

\begin{document}
\maketitle

\section{Problem 1} Prove that the num. of partitions of a positive int $n$ into $k$ \textit{even} parts 
is eq. to the num. of partitions of $n$ into $k$ odd parts.

\begin{itemize}
  \item \textbf{Idea:} First, express $n$ in two ways by decomposing it into $k$ even parts and into $k$ odd parts. We can write 
  \begin{align*}
    n &= 2a_1 + 2a_2 + \cdots + 2a_k,\\
    n - k &= 2a_1 + 2a_2 + \cdots + 2a_k - k,\\
    n - k &= (2a_1 - 1) + (2a_2 - 1) + \cdots + (2a_k - 1)\,.
  \end{align*}
  The first line expresses $n$ as a sum of $k$ even numbers (each $2a_i$). Subtracting $k$ from both sides shows that $n-k$ can be expressed as a sum of $k$ odd numbers (each $2a_i - 1$). 
  
  From this idea, we expect a correspondence between partitions of $n$ into $k$ even parts and partitions of $n-k$ into $k$ odd parts.
  
  \item \textbf{Formal proof:} Define the sets 
  \[
    P_e(n,k) := \{\text{partitions of $n$ into $k$ even parts}\}, 
  \] 
  and 
  \[
    P_o(n,k) := \{\text{partitions of $n$ into $k$ odd parts}\}. 
  \] 

  
  For $n$ even, it turns out these two sets have the same size: 
  \[
    |P_e(n,k)| = |P_o(n-k,k)|\,.
  \] 
  To show this, we construct an explicit bijection. Define a function 
  \[
    f: P_e(n,k) \to P_o(n-k,k)\,,
  \] 
  by 
  \[
    f(b_1,b_2,\ldots,b_k) = (\,b_1 - 1,\; b_2 - 1,\; \ldots,\; b_k - 1\,)\,. 
  \] 
  Here $(b_1,\ldots,b_k) \in P_e(n,k)$ means $b_1 + \cdots + b_k = n$ with each $b_i$ even, so $b_i \ge 2$ for all $i$. Thus, each $b_i - 1 \ge 1$ and is odd, which ensures $f(b_1,\ldots,b_k) \in P_o(n-k,k)$ is a partition of $n-k$ into $k$ odd parts.

  The function $f$ is well-defined. Moreover, it is invertible by simply adding 1 to each part. In fact, define 
  \[
    g: P_o(n-k,k) \to P_e(n,k)
  \] 
  as 
  \[
    g(c_1,c_2,\ldots,c_k) = (\,c_1 + 1,\; c_2 + 1,\; \ldots,\; c_k + 1\,)\,. 
  \] 
  If $(c_1,\ldots,c_k)$ is a partition of $n-k$ into odd parts, then each $c_i$ is odd and $c_i \ge 1$, so $c_i + 1$ is even and $\ge 2$, and $\sum_{i=1}^k (c_i + 1) = (n-k) + k = n$. Thus $g(c_1,\ldots,c_k) \in P_e(n,k)$. It is easy to check that $g$ is indeed the inverse of $f$: we have $f(g(c_1,\ldots,c_k)) = (c_1,\ldots,c_k)$ and $g(f(b_1,\ldots,b_k)) = (b_1,\ldots,b_k)$. Therefore, $f$ is bijective.
\end{itemize}

\vspace{1em} % a little vertical space between tasks for clarity

\section{Problem 2} Show that the number of partitions of a positive int $n$ with at most $k$ components is eq. to the num. of partitions of $2n$ with at most $k$ even components.

\begin{itemize}
  \item \textbf{Idea:} We consider partitions of $n$ that have at most $k$ parts. Let 
  \[
    n = a_1 + a_2 + \cdots + a_L,
  \] 
  where $L \le k$ and $a_1 \ge a_2 \ge \cdots \ge a_L > 0$. (In other words, $a_1,\ldots,a_L$ are the parts of a partition of $n$, listed in non-increasing order, with at most $k$ parts.) For example, if $n=5$ and $k=3$, the partitions of 5 with at most 3 parts can be represented (padding with zeros up to 3 parts) as:
  \[
    (2,2,1), \qquad (3,2,0), \qquad (5,0,0)\,,
  \] 
  where we use $0$ to indicate an empty part (no number in that position).
  
  Notice that doubling each part in these examples produces a partition of $2n=10$ with only even parts (and still at most 3 components). For instance, $(2,2,1)$ doubles to $(4,4,2)$, $(3,2,0)$ doubles to $(6,4,0)$, and $(5,0,0)$ doubles to $(10,0,0)$. This suggests a direct correspondence between partitions of $n$ (up to $k$ parts) and partitions of $2n$ into even parts (up to $k$ parts).
  
  \item \textbf{Formal proof:} Let us define the relevant sets in words (as suggested in the notes): 
  \begin{itemize}
    \item $P(n,\le k)$ — the set of all partitions of $n$ with \emph{at most} $k$ components (parts).
    \item $P_e(2n,\le k)$ — the set of all partitions of $2n$ with at most $k$ \emph{even} components.
  \end{itemize}
  We aim to show that 
  \[
    |P(n,\le k)| = |P_e(2n,\le k)|\,,
  \] 
  i.e. the two sets have equal cardinality. To prove this, we construct a bijection $f: P(n,\le k) \to P_e(2n,\le k)$. Given any partition $(a_1,a_2,\dots,a_L)$ of $n$ (with $L \le k$), map it to 
  \[
    f(a_1,a_2,\ldots,a_L) = (\,2a_1,\; 2a_2,\; \ldots,\; 2a_L\,)\,. 
  \] 
  In other words, $f$ doubles each part of the partition of $n$. If the partition of $n$ has fewer than $k$ parts, we may imagine that it is padded with zeros (as above) which double to zeros, so the resulting partition of $2n$ still has at most $k$ parts. By construction, $f(a_1,\ldots,a_L)$ is a partition of $2n$ in which every part is even, so indeed $f(a_1,\ldots,a_L) \in P_e(2n,\le k)$. The function $f$ is invertible by halving each even part: for any partition $(b_1,b_2,\ldots,b_M) \in P_e(2n,\le k)$ (each $b_i$ even), the inverse map $f^{-1}$ gives 
  \[
    f^{-1}(b_1,b_2,\ldots,b_M) = \Big(\frac{b_1}{2},\; \frac{b_2}{2},\; \ldots,\; \frac{b_M}{2}\Big)\,,
  \] 
  which is a partition of $2n/2 = n$ with at most $k$ parts. Thus $f$ is a bijection between $P(n,\le k)$ and $P_e(2n,\le k)$, and consequently $|P(n,\le k)| = |P_e(2n,\le k)|$.
\end{itemize}

\end{document}

