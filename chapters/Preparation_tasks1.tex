\documentclass[docmute]{article}

% AMS packages for enhanced math typesetting and symbols:
\usepackage{amsmath}  % Provides enhanced math features like align, gather, etc.
\usepackage{amssymb}  % Provides additional math symbols
\usepackage{amsthm}   % Enables theorem-like environments

\usepackage{graphicx} % Provides commands for including and managing external graphics (e.g., figures, images)

% Package for customizing list environments:
\usepackage{enumitem} % Allows control over layout of lists (itemize, enumerate, etc.)

% Full-page layout package:
\usepackage{fullpage} % Uses more of the page area by reducing margins

% TikZ package for drawing graphics:
\usepackage{tikz}     % Used for creating high-quality diagrams and figures

% Microtype package for typographical enhancements:
\usepackage{microtype} % Improves justification, kerning, and overall appearance

% Package for typesetting polynomials:
\usepackage{polynom}  % Provides commands for polynomial long division and related tasks

% Package for controlling figure placement:
\usepackage{placeins} % Provides the \FloatBarrier command to control floating environments

% Forest package for drawing trees:
\usepackage{forest}   % Simplifies the creation of tree diagrams

% Package to allow one LaTeX file to input another:
\usepackage{docmute}  % Allows this file to be included in another document without reloading the preamble

% Load additional TikZ libraries:
\usetikzlibrary{trees} % Provides additional tree-specific commands for TikZ

% Define theorem-like environments using amsthm:
\newtheorem{corollary}{Corollary} % Defines a new "corollary" environment
\newtheorem{lemma}{Lemma}         % Defines a new "lemma" environment


\input{watermark/watermark.tex}


\title{Preparation Tasks 1}
\author{Tomasz Brengos \\  
Committers : Aliaksei Kudzelka}
\date{}

\begin{document}
\maketitle

\section{Task 1}

\textbf{Problem Statement:}

There are 20 students and 5 trips. In each of the following scenarios, count the number of ways the students can be assigned to trips.

\begin{enumerate}
  \item[(a)] \emph{Students are different, trips are different.} (I.e., we do care exactly which student goes to which distinct trip.)
  \item[(b)] \emph{Trips are different, but we only care about how many students go to each trip.} (Not which specific students, only the counts per trip.)
  \item[(c)] \emph{Trips are not distinguished, and we only care about the number of students in each trip.} (All trips are identical, so only the final ``multiset of group sizes'' matters.)
  \item[(d)] \emph{We do not care who goes where, but only with whom they go.} (We only care about the partition of the 20 students into some grouping, ignoring which trip is which.)
  \item[(e)] \emph{Students are different, trips are different, and each trip has the same number of students.}
\end{enumerate}

\subsection*{Solutions and Explanations}

\paragraph{(a)} 
Each of the 20 \textbf{distinct} students independently chooses one of 5 \textbf{distinct} trips. This gives
\[
5^{20}
\]
total ways (each student has 5 choices).

\paragraph{(b)} 
Now the students’ individual identities no longer matter; we only care that, for example, “Trip 1 has 7 students, Trip 2 has 3 students,” etc. Since there are 5 \textbf{distinct} trips, we want the number of 5-tuples \((n_1, n_2, n_3, n_4, n_5)\) of nonnegative integers summing to 20:
\[
n_1 + n_2 + n_3 + n_4 + n_5 = 20.
\]
By the stars-and-bars formula, the count is
\[
\binom{20 + 5 - 1}{5 - 1} \;=\; \binom{24}{4}.
\]

\paragraph{(c)}
Now the 5 trips themselves are \textbf{indistinguishable}, and we only care about how many students end up in each group (not which trip is which). Conceptually, this is the number of ways to partition 20 (distinct) students into up to 5 unlabeled subsets.

Interpreted as integer partitions, we want the number of partitions of the integer 20 into at most 5 parts. Denote by \(p(n, i)\) the number of ways to partition \(n\) into exactly \(i\) (positive) parts. Then the total number of partitions of 20 into at most 5 parts is

\[
p_{\le 5}(20) \;=\; \sum_{i=1}^{5} p(20, i).
\]


\paragraph{(d)}
We do not care who goes where, but only with whom they go. That is, we only care about how to group the 20 students, disregarding which trip label is attached to each group.

In terms of set partitions, the total number of ways to partition \(20\) distinct students into any number of unlabeled subsets is the Bell number \(B_{20}\). The Bell number can be expressed as a sum of Stirling numbers of the second kind:
\[
B_{20} 
\;=\; 
\sum_{k=0}^{20} S(20, k),
\]
where \(S(n, k)\) (with \(S(n, 0) = 0\) for \(n>0\)) counts the number of ways to partition \(n\) distinct elements into exactly \(k\) nonempty unlabeled subsets. Equivalently, one can start the sum at \(k=1\) since there are no partitions of 20 elements into 0 subsets:
\[
B_{20} 
\;=\; 
\sum_{k=1}^{20} S(20, k).
\]

\paragraph{(e)} 
Students are different, trips are different, and each trip has the \emph{same} number of students. Since \(20\) is divisible by \(5\), that means exactly 4 students must go on each trip. Count the ways to split 20 distinct students into 5 distinct groups of 4 each. One way to see this is:

\[
\underbrace{\binom{20}{4}}_{\text{Trip 1}} \;
\underbrace{\binom{16}{4}}_{\text{Trip 2}} \; 
\cdots 
\underbrace{\binom{4}{4}}_{\text{Trip 5}}.
\]
Equivalently, the multinomial coefficient
\[
\frac{20!}{4!\,4!\,4!\,4!\,4!}.
\]

\bigskip

\section{Task 2}

\textbf{Problem Statement:}

We have 6 toy cars and 4 dolls (total 10 toys), and 10 boxes. In each scenario, count how many ways there are to place the 10 toys into the 10 boxes, under various assumptions of distinctness or identicalness:

\begin{enumerate}
  \item[(a)] Everything (cars, dolls, boxes) is different.
  \item[(b)] Everything is different and every toy goes into a separate box.
  \item[(c)] All toys are different, but boxes are identical.
  \item[(d)] Cars are different, but all dolls are identical; boxes are different.
  \item[(e)] All cars are identical, all dolls are identical; boxes are different.
  \item[(f)] Cars are different, but all dolls are identical; boxes are identical.
  \item[(g)] We don’t care which toy is which, but only how many toys are in each box; boxes are different.
  \item[(h)] We don’t care which toy is which, but only how many toys are in each box; boxes are identical.
\end{enumerate}

\subsection*{Solutions and Explanations}

Let us denote our toys as follows:
\[
\text{Cars}: C_1, C_2, C_3, C_4, C_5, C_6; 
\quad
\text{Dolls}: D_1, D_2, D_3, D_4.
\]
We have 10 boxes, say \(B_1, B_2, \dots, B_{10}\).

\paragraph{(a) Everything is different.}
Each of the 10 distinct toys can go into any of 10 distinct boxes. Hence there are 
\[
10^{10}
\]
ways (each toy has 10 choices independently).

\paragraph{(b) Everything is different and every toy goes into a separate box.}
We must place exactly one of the 10 distinct toys in each of the 10 distinct boxes (no box is left empty, no box has more than one toy). This is simply a permutation of 10 objects into 10 boxes:
\[
10! 
\]
ways.

\paragraph{(c) All toys are different, but boxes are identical.}
We have 10 \emph{distinct} toys (labeled objects) to be placed into up to 10 \emph{indistinguishable} boxes. Equivalently, this is the number of ways to partition a set of 10 labeled elements into any number of unlabeled subsets (possibly fewer than 10 if some boxes are empty).

The total count is the Bell number \(B_{10}\). It can be expressed as a sum of Stirling numbers of the second kind:
\[
B_{10}
\;=\;
\sum_{k=0}^{10} S(10,k)
\;=\;
\sum_{k=1}^{10} S(10,k),
\]
where \(S(n,k)\) is the number of ways to partition \(n\) distinct objects into \(k\) nonempty subsets. There is no simple closed form for \(B_{10}\), so we typically leave the answer in this summation form or as \(B_{10}\) itself.

\paragraph{(d) Cars are different, but dolls are identical; boxes are different.}
- We have 6 distinct cars \(C_1,\ldots,C_6\). Each can go into any of 10 distinct boxes: \(10^6\) ways.
- We have 4 \emph{identical} dolls. Distributing 4 indistinguishable objects into 10 distinct boxes is given by the ``stars-and-bars'' formula:
  \[
  \binom{4 + 10 - 1}{4} \;=\;\binom{13}{4}.
  \]
Multiply these independent choices:
\[
10^6 \;\times\; \binom{13}{4}.
\]

\paragraph{(e) All cars are identical, all dolls are identical; boxes are different.}
- Distribute 6 identical cars into 10 distinct boxes: \(\displaystyle \binom{6 + 10 - 1}{6} = \binom{15}{6}.\)
- Distribute 4 identical dolls into 10 distinct boxes: \(\displaystyle \binom{4 + 10 - 1}{4} = \binom{13}{4}.\)
Multiply for the total:
\[
\binom{15}{6} \;\times\; \binom{13}{4}.
\]

\paragraph{(f)}

We have a total of \(10\) toys:
\[
\underbrace{C_1, C_2, C_3, C_4, C_5, C_6}_{\text{6 distinct cars}}
\quad\text{and}\quad
\underbrace{D, D, D, D}_{\text{4 identical dolls}},
\]
which are to be placed into \(10\) identical boxes (i.e.\ unlabeled subsets). 
We want the number of different ways to partition these toys according to which cars appear together and how many dolls join them.

\subparagraph*{Step 1: Partition the Cars}

First, partition the \(6\) \emph{distinct} cars into \(k\) nonempty subsets (blocks).  
The number of ways to do this is given by the Stirling number of the second kind 
\[
S(6,k).
\]
Since each of the \(6\) cars is distinct, we have 
\(S(6,1)\) ways to partition them into \(1\) block, 
\(S(6,2)\) ways into \(2\) blocks, \(\dots\), up to 
\(S(6,6)\) ways to put each car in its own block.  
Hence \(k\) ranges over \(1 \le k \le 6\).

\subparagraph*{Step 2: Distribute the Dolls}

Next, we need to distribute the \(4\) \emph{identical} dolls among these \(k\) car-blocks 
\emph{and possibly place some (or all) dolls in new subsets} that contain only dolls.

Let
\[
a = \text{the number of dolls placed into the }k\text{ car-containing blocks},
\]
so that the remaining \(4-a\) dolls may form additional subsets, each of which contains \emph{only} dolls (with no cars).

\subparagraph{Dolls Within the Existing \(k\) Blocks.}

Since the \(k\) blocks of cars are \emph{distinct} (once the cars have been chosen to form those blocks), distributing \(a\) identical dolls among the \(k\) blocks is counted by the usual ``stars and bars'' formula:
\[
\binom{a + k - 1}{k - 1}.
\]

\subparagraph{New Subsets of Only Dolls.}

Now, out of the \((4 - a)\) remaining dolls, we can create any number \(t\) of subsets \((0 \le t \le 4-a)\), each consisting of a positive number of dolls.  
Because these new subsets (boxes) are \emph{unlabeled}, the number of ways to partition \((4 - a)\) identical dolls into exactly \(t\) nonempty unlabeled subsets is
\[
p(4 - a, t),
\]
where \(p(n, k)\) denotes the number of ways to partition \(n\) identical items into \(k\) nonempty unlabeled subsets.  
Summing over all \(t\) from \(0\) to \((4-a)\) then gives the total ways to form any number of doll-only subsets out of the \((4 - a)\) dolls.

\subparagraph{Putting It All Together}

For a fixed value of \(k\) (the number of car-blocks), the number of ways to handle the dolls is
\[
\sum_{a=0}^{4}
\Biggl[
  \binom{a + k - 1}{k - 1}
  \;\times\;
  \sum_{t=0}^{4 - a} p(4 - a, t)
\Biggr].
\]
We then multiply by \(S(6,k)\), the ways to choose the \(k\) car-blocks in the first place.  
Finally, we sum over all possible \(k\):
\[
\text{Total ways}
\;=\;
\sum_{k=1}^{6}
\biggl[
  S(6,k)
  \times
  \sum_{a=0}^{4}\!\Bigl(
      \binom{a + k - 1}{k - 1}
      \;\times\;
      \sum_{t=0}^{4-a} p(4 - a, t)
  \Bigr)
\biggr].
\]
\[
\textbf{``However, this task should not appear on the test.''}
\]

\paragraph{(g) We don’t care which toy is which, only about how many toys are in each box; boxes are different.}
Now all 10 toys are treated as identical. With 10 \emph{distinct} boxes, we only need the distribution of 10 identical items into 10 distinct bins, i.e.\ the number of solutions to
\[
n_1 + n_2 + \dots + n_{10} = 10
\]
where \(n_i \ge 0\). By stars-and-bars, that is
\[
\binom{10 + 10 - 1}{10} \;=\;\binom{19}{10}.
\]

\paragraph{(h) We don’t care which toy is which, only about how many toys are in each box; boxes are identical.}
Now both the toys and the boxes are considered identical. We want the number of ways to partition 10 identical objects among up to 10 identical boxes. Equivalently, this is the number of ways to express the integer 10 as a sum of positive integers, ignoring order. 

We denote by \(p(n)\) the total number of partitions of \(n\). This can also be expressed in terms of \(p(n,k)\), which is the number of ways to partition \(n\) into exactly \(k\) positive parts:
\[
p(n) \;=\; \sum_{k=1}^{n} p(n,k).
\]
Since we can’t have more than 10 parts if each part is positive, for \(n=10\) we write
\[
p(10) 
\;=\;
\sum_{k=1}^{10} p(10,k).
\]
It is known (by direct enumeration or from tables) that
\[
p(10) \;=\; 42.
\]
Hence, there are 42 ways to distribute 10 indistinguishable toys among 10 indistinguishable boxes.

\bigskip
\noindent
\textbf{Remark.} Parts like (f) are more intricate when combining “partially identical” toys with “identical boxes.” Typically, such problems require detailed case analysis or generating functions.

\section{Task 3}

There are \(n\) rows with \(n\) chairs in each row (so \(n^2\) chairs in total), and we have \(k\) students (where \(k \le n\)) who are all distinct unless otherwise noted. We want to count the number of ways they can sit subject to various conditions. In each part, “at most one person in each chair” is always assumed.

\begin{enumerate}
  \item[\textbf{(a)}]
    \textbf{No extra assumptions.}

    We are simply placing \(k\) \emph{distinct} students into \(n^2\) distinct seats. First choose which \(k\) seats will be occupied, then permute the \(k\) students among those chosen seats. The number of ways is:
    \[
      \binom{n^2}{k}\,k!
      \quad=\quad \frac{n^2!}{(n^2 - k)!}.
    \]

  \item[\textbf{(b)}]
    \textbf{Everyone sits in the first row.}

    The first row has \(n\) chairs and \(k \le n\) students. We must pick which \(k\) of the \(n\) seats are used, and then assign the \(k\) distinct students to those seats. Hence
    \[
      \binom{n}{k}\,k!.
    \]

  \item[\textbf{(c)}]
    \textbf{We only care about the \emph{numbers} of students sitting in each row.}

    Now we ignore which particular seats in each row are used \emph{and} which particular students go to a given row.  We only record the integer vector \((x_1, x_2,\dots, x_n)\) of row‐occupancies, where \(x_i\) is the number of students in row~\(i\), and
    \[
      x_1 + x_2 + \cdots + x_n \;=\; k,
      \quad 0 \,\le\, x_i \,\le\, n.
    \]
    Since \(k \le n\), the constraint \(x_i \le n\) is automatically satisfied.  The number of nonnegative solutions to \(x_1 + \dots + x_n = k\) is
    \[
      \binom{k + n - 1}{n - 1}.
    \]
    (Here the students themselves are no longer distinguished, nor are seats in the same row.)

  \item[\textbf{(d)}]
    \textbf{We only care about \emph{which chairs} are taken, not who sits in them.}

    In this scenario, we ignore the identity of students and only note \emph{the set of occupied seats} (of size~\(k\)).  Hence we need to choose exactly \(k\) chairs out of \(n^2\), with no regard to which student goes where.  The count is
    \[
      \binom{n^2}{k}.
    \]

    \item[\textbf{(e)}]
    \textbf{We only care about the numbers of students in each row, with each row having at most as many students as the previous row.}
    
    Let \(x_i\) be the number of students in row~\(i\). Then we record
    \[
      (x_1,x_2,\ldots,x_n)
      \quad\text{such that}\quad
      x_1 \ge x_2 \ge \cdots \ge x_n \ge 0
      \quad\text{and}\quad
      x_1 + x_2 + \cdots + x_n = k.
    \]
    Equivalently, we are looking for a partition of \(k\) into at most \(n\) parts. Let
    \[
      p(k,j)
    \]
    be the number of partitions of \(k\) into exactly \(j\) positive parts. Then the number of partitions of \(k\) into \emph{at most} \(n\) parts is
    \[
      p_{\le n}(k)
      \;=\;
      \sum_{j=0}^n p(k,j).
    \]
    (Of course, \(p(k,0)=0\) unless \(k=0\), but we often include \(j=0\) for completeness.)
    
    Hence the number of ways to seat \(k\) students under these conditions is
    \[
      \boxed{
        p_{\le n}(k) \;=\;\sum_{j=0}^n p(k,j).
      }
    \]
    There is no simpler closed‐form expression for these partition numbers, but tables and generating functions can be used to compute them for specific \(k\) and \(n\). 
    

    \item[\textbf{(f)}]
    \textbf{We do not care who sits where, only which students sit \emph{together in the same row} (no matter which row).}
    
    In this case, the arrangement is determined solely by partitioning the set of \(k\) distinct students into nonempty subsets, where each subset corresponds to a group sitting together in one row (and the rows themselves are unlabeled). The number of ways to partition a \(k\)-element set into \(j\) nonempty subsets is given by the Stirling number of the second kind, denoted \(S(k,j)\). Therefore, if we allow any number \(j\) of groups (with \(1 \le j \le k\)), the total number of ways is:
    \[
    \sum_{j=1}^{k} S(k,j).
    \]
    This sum is known as the Bell number \(B_k\); that is,
    \[
    B_k = \sum_{j=1}^{k} S(k,j).
    \]
    Thus, the number of ways is:
    \[
    \boxed{B_k.}
    \]
    
    \item[\textbf{(f)}]

    We have a total of \(10\) toys:
    
    \[
    \underbrace{C_1, C_2, C_3, C_4, C_5, C_6}_{\text{6 distinct cars}}
    \quad\text{and}\quad
    \underbrace{D, D, D, D}_{\text{4 identical dolls}},
    \]
    which are to be placed into \(10\) identical boxes (i.e.\ unlabeled subsets). 
    We want the number of different ways to partition these toys according to which cars appear together and how many dolls join them.
    
    \subsection*{Step 1: Partition the Cars}
    
    First, partition the \(6\) \emph{distinct} cars into \(k\) nonempty subsets (blocks).  The number of ways to do this is given by the Stirling number of the second kind 
    \[
    S(6,k).
    \]
    Since each of the \(6\) cars is distinct, we have \(S(6,1)\) ways to partition them into \(1\) block, \(S(6,2)\) ways into \(2\) blocks, \(\dots\), up to \(S(6,6)\) ways to put each car in its own block.  Hence \(k\) ranges over \(1 \le k \le 6\).
    
    \subsection*{Step 2: Distribute the Dolls}
    
    Next, we need to distribute the \(4\) \emph{identical} dolls among these \(k\) car-blocks \emph{and possibly place some (or all) dolls in new subsets} that contain only dolls. 
    
    Let
    \[
    a = \text{the number of dolls placed into the }k\text{ car-containing blocks},
    \]
    so that the remaining \(4-a\) dolls may form additional subsets, each of which contains \emph{only} dolls (with no cars).
    
    \paragraph{Dolls Within the Existing \(k\) Blocks.}
    
    Since the \(k\) blocks of cars are \emph{distinct} (once the cars have been chosen to form those blocks), distributing \(a\) identical dolls among the \(k\) blocks is counted by the usual ``stars and bars'' formula:
    \[
    \binom{a + k - 1}{k - 1}.
    \]
    
    \paragraph{New Subsets of Only Dolls.}
    
    Now, out of the \((4 - a)\) remaining dolls, we can create any number \(t\) of subsets (\(0 \le t \le 4-a\)), each subset consisting of a positive number of dolls.  Because these new subsets (boxes) are \emph{unlabeled}, the number of ways to partition \((4 - a)\) identical dolls into exactly \(t\) positive parts is
    \[
    p_t(4 - a),
    \]
    where \(p_t(n)\) denotes the number of ways to partition \(n\) identical items into \(t\) nonempty unlabeled subsets. Summing over all \(t\) from \(0\) to \((4-a)\) then gives the total ways to form any number of doll-only subsets out of the \((4 - a)\) dolls.
    
    \subsection*{Putting It All Together}
    
    For a fixed value of \(k\) (the number of car-blocks), the number of ways to handle the dolls is
    \[
    \sum_{a=0}^{4}
    \Biggl[
      \binom{a + k - 1}{k - 1}
      \;\times\;
      \sum_{t=0}^{4 - a} p_{t}(4 - a)
    \Biggr].
    \]
    
    We then multiply by \(S(6,k)\), the ways to choose the \(k\) car-blocks in the first place.  Finally, we sum over all possible \(k\):
    \[
    \text{Total ways}
    \;=\;
    \sum_{k=1}^{6}
    \biggl[
      S(6,k)
      \times
      \sum_{a=0}^{4}\!\Bigl(
          \binom{a + k - 1}{k - 1}
          \;\times\;
          \sum_{t=0}^{4-a} p_{t}(4-a)
      \Bigr)
    \biggr].
    \]
    
    A direct computation (by hand or program) yields the final count
    \[
    \boxed{13{,}960}.
    \]
    
    Hence, there are \(\boxed{13{,}960}\) ways to place the 6 distinct cars and 4 identical dolls into 10 identical boxes.
    
  \item[\textbf{(g)}]
    \textbf{Everyone sits in the first 5 rows, and each of these 5 rows has the \emph{same} number of students (assume $5\mid k$).}

    Let \(k=5m\).  We have 5 labeled rows (rows~1 through~5), and each must have exactly \(m\) students.  We do care about \emph{which} students go to each row, and also which seats they occupy (and the students are distinct).  The counting proceeds in two stages:

    \begin{enumerate}
      \item Distribute the $k$ students into 5 labeled groups of size $m$ each:
        \[
          \binom{k}{m}\,\binom{k-m}{m}\,\cdots = \frac{k!}{(m!)^5}.
        \]
      \item For each group of $m$ destined for row~$i$, choose which $m$ seats (out of $n$ in that row) they occupy, and permute the $m$ distinct students among those $m$ seats:
        \[
          \binom{n}{m}\,m! \quad\text{ways per row}.
        \]
      Hence, for 5 rows, multiply $\bigl(\binom{n}{m} m!\bigr)^5$.

    \end{enumerate}

    Overall, the number of ways is:
    \[
      \frac{k!}{(m!)^5} \times \Bigl(\binom{n}{m}\,m!\Bigr)^5
      \;=\;
      k!\,\biggl[\binom{n}{m}\biggr]^5,
      \quad\text{where }m=\frac{k}{5}.
    \]

  \item[\textbf{(h)}]
    \textbf{Everyone sits in the first 3 rows; we do not care which physical seats are used, but we \emph{do} care about left‐right adjacency (ignoring empty chairs).}

    Here each row is treated as a consecutive \emph{line} of however many students it has.  We do care which student is to the left or right of which other student in that row, but we ignore the actual seat numbers and any gaps.  Thus, for row~\(i\) having \(x_i\) students, the $x_i$ chosen students can be arranged in \((x_i)!\) ways (linear order).  Then we sum over all ways to split $k$ distinct students into 3 labeled groups (for the 3 rows) and order each group:

    \begin{enumerate}
      \item Choose a triple \((x_1,x_2,x_3)\) with $x_1+x_2+x_3 = k$.
      \item Choose which $x_1$ students go to row~1, $x_2$ to row~2, etc.  That is a multinomial factor $\frac{k!}{x_1!\,x_2!\,x_3!}$.
      \item Permute each group internally in $(x_i)!$ ways.

      But multiplying $\frac{k!}{x_1!\,x_2!\,x_3!}\times (x_1)!(x_2)!(x_3)! = k!.$
    \end{enumerate}

    Finally, we sum over all nonnegative $(x_1,x_2,x_3)$ summing to $k$.  The number of such triples is $\binom{k+3-1}{3-1} = \binom{k+2}{2}.$  Thus total seatings:
    \[
      k!\,\binom{k+2}{2}.
    \]

  \item[\textbf{(i)}]
    \textbf{If a row is not empty, it contains \emph{at least two} people; we do not distinguish students \emph{and} we do not distinguish chairs in the same row.}

    So we only care how many students are in each row, with $x_i=0$ or $x_i\ge2$.  Since $k\le n$, it is possible that many rows are empty.  We want nonnegative $x_1,\dots,x_n$ with $x_1+\dots+x_n = k$ and each nonempty $x_i\ge2$.  Since we do \emph{not} distinguish which particular students go to row~$i$, nor which seats they occupy, we are effectively counting the integer solutions:

    \[
      x_i \in \{0\}\cup \{2,3,\dots\}, 
      \quad x_1+\dots+x_n = k.
    \]
    Let $m$ be the number of nonempty rows.  Then $m$ satisfies $1 \le m \le \lfloor k/2 \rfloor$, and we choose which $m$ rows are nonempty in $\binom{n}{m}$ ways.  In each of those $m$ chosen rows, let $x_i \ge2$.  Set $y_i = x_i - 2 \ge 0$, so $\sum_{i=1}^m y_i = k - 2m.$  The number of nonnegative solutions to that is $\binom{(k-2m) + m -1}{m-1} = \binom{k - m -1}{m-1}$, valid as long as $k \ge 2m.$  

    Summing over all $m$ gives
    \[
      \sum_{m=1}^{\lfloor k/2 \rfloor}
        \binom{n}{m}\,\binom{k - m - 1}{m-1},
      \quad
      \text{(assuming $k>0$).}
    \]

  \item[\textbf{(j)}]
    \textbf{If a row is not empty, it has at least two people; we do not distinguish students, \emph{but} we do care which specific seats are occupied.}



  \item[\textbf{(k)}]
    \textbf{If a chair is empty, then all chairs \emph{to its left in the same row} are empty as well.}

    Equivalently, in each row that has $x_i>0$ seats occupied, those must be the leftmost $x_i$ chairs of that row, with no gaps.  If we do \emph{not} distinguish students and only care about which seats are occupied, then row~$i$ is either completely empty ($x_i=0$) or we occupy seats 1 through $x_i$.  Summing over all rows, we must choose a total of $k$ occupied chairs:

    \[
      x_1 + x_2 + \cdots + x_n = k,
      \quad 0 \le x_i \le n.
    \]
    Since $k \le n$, the row capacity $x_i \le n$ is never an issue.  The number of nonnegative integer solutions is the standard stars‐and‐bars count:
    \[
      \binom{k + n - 1}{n - 1}.
    \]
    Thus there are $\binom{k + n - 1}{n - 1}$ ways to pick exactly which seats are occupied (each row’s occupied block starts at seat~1).

\end{enumerate}

\end{document}