\documentclass{article}
\usepackage{amsmath, amssymb, amsthm, enumitem,fullpage,tikz,microtype,polynom,placeins,forest}
\usepackage{docmute}
\newtheorem{corollary}{Corollary}
\newtheorem{lemma}{Lemma}
\usetikzlibrary{trees}

\title{Functions between sets}
\author{Tomasz Brengos \\  
Committers : ???}
\date{}

\begin{document}
\maketitle
Let $N$ and $R$ be sets with $|N| = n$ and $|R| = r$.

\begin{enumerate}[label=(\roman*)]
    \item \textbf{Total Functions:}  
    The number of functions from \( N \) to \( R \) is  
    \[
    r^n.
    \]  
    Explanation: For every element in \( N \), there are \( |R| = r \) possible values in \( R \). Thus, for the first element, there are \( r \) choices, for the second element, there are \( r \) choices, and so on.  
    Applying the rule of product, the total number of functions is \( r^n \).

    \item \textbf{Injective Functions:}  
    When \( r \geq n \), an injective function (one-to-one) from \( N \) to \( R \) can be chosen by assigning distinct images to the \( n \) elements.  

    If a function is injective, then for each value in the range there is only one corresponding argument. This means that function values cannot repeat, ensuring that \( x_1 \neq x_2 \) implies \( f(x_1) \neq f(x_2) \).  

    Since there are \( |R| = r \) choices for the first argument, \( r-1 \) choices for the second, \( r-2 \) for the third, and so on, applying the rule of product, the number of injective functions from \( N \) to \( R \) is:  
    \[
    r \cdot (r-1) \cdots (r-n+1) = \frac{r!}{(r-n)!}.
    \]
    \item \textbf{Surjective Functions:}  
    A function is surjective (onto) if every element in \( R \) has a pre-image in \( N \), meaning every element in \( R \) is an image of some element in \( N \).  
    Consider a surjection \( f: N \to R = \{y_1, y_2, \dots, y_r\} \). We observe that the preimages \( f^{-1}(y_1), f^{-1}(y_2), \dots, f^{-1}(y_r) \) form a partition of \( N \) into \( r \) non-empty subsets, as each element \( y_i \) in \( R \) corresponds to one or more elements from \( N \).  
    The number of ways to partition \( N \) into \( r \) parts is given by the Stirling number \( S(n, r) \), and since we can permute the \( r \) elements in \( R \) in \( r! \) ways, the total number of surjective functions from \( N \) to \( R \) is:  
    \[
    r! \, S(n, r),
    \]
    where \( S(n, r) \) is the Stirling number of the second kind, counting the ways to partition \( N \) into \( r \) non-empty subsets.

\end{enumerate}

\paragraph{Example:}  
For \( N = \{1, 2, 3\} \) and \( R = \{y_1, y_2\} \):
\begin{itemize}[nosep]
    \item[] Here \( |N| = 3 \) and \( |R| = 2 \).
    \item Total functions: $2^3 = 8$.
    \item Injective functions: Not possible since $|R|<|N|$.
    \item Surjective functions: Consider all possible surjective functions:

 \( f_1: \{1, 2\} \mapsto y_1, 3 \mapsto y_2 \)
- Another possible permutation for this partition: \( f_2: \{1, 2\} \mapsto y_2, 3 \mapsto y_1 \)
 \( f_3: \{2, 3\} \mapsto y_1, 1 \mapsto y_2 \)
- Another possible permutation for this partition: \( f_4: 1 \mapsto y_1, \{2, 3\} \mapsto y_2 \)
 \( f_5: \{1, 3\} \mapsto y_1, 2 \mapsto y_2 \)
- Another possible permutation for this partition: \( f_6: 2 \mapsto y_1, \{1, 3\} \mapsto y_2 \)

So, we have 6 surjective functions. Using the formula for surjective functions, we first find the Stirling number \( S(3, 2) = 3 \), which corresponds to the number of partitions without considering permutations. Then, accounting for the permutations of the \( r = 2 \) elements in \( R \), we compute:
\[
2! \cdot S(3, 2) = 2! \cdot 3 = 6,
\]
which matches the number of surjective functions we listed.
\end{itemize}
\end{document}